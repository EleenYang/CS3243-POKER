%%%% ijcai18.tex

\typeout{IJCAI-18 Instructions for Authors}

% These are the instructions for authors for IJCAI-18.
% They are the same as the ones for IJCAI-11 with superficical wording
%   changes only.

\documentclass{article}
\pdfpagewidth=8.5in
\pdfpageheight=11in
% The file ijcai18.sty is the style file for IJCAI-18 (same as ijcai08.sty).
\usepackage{ijcai18}

% Use the postscript times font!
\usepackage{times}
\usepackage{xcolor}
\usepackage{soul}
\usepackage[utf8]{inputenc}
\usepackage[small]{caption}

% the following package is optional:
%\usepackage{latexsym} 

% Following comment is from ijcai97-submit.tex:
% The preparation of these files was supported by Schlumberger Palo Alto
% Research, AT\&T Bell Laboratories, and Morgan Kaufmann Publishers.
% Shirley Jowell, of Morgan Kaufmann Publishers, and Peter F.
% Patel-Schneider, of AT\&T Bell Laboratories collaborated on their
% preparation.

% These instructions can be modified and used in other conferences as long
% as credit to the authors and supporting agencies is retained, this notice
% is not changed, and further modification or reuse is not restricted.
% Neither Shirley Jowell nor Peter F. Patel-Schneider can be listed as
% contacts for providing assistance without their prior permission.

% To use for other conferences, change references to files and the
% conference appropriate and use other authors, contacts, publishers, and
% organizations.
% Also change the deadline and address for returning papers and the length and
% page charge instructions.
% Put where the files are available in the appropriate places.

\title{Poker Project - Team 15}


% Multiple author syntax (remove the single-author syntax above and the \iffalse ... \fi here)

\author{
Gong Changda, Kong Zijin, Liu Yingnan, Liu Yiyang, Yang Sihan
\\ 
National University of Singapore\\
%
first,second,third,fourth@u.nus.edu
}
% If your authors do not fit in the default space, you can increase it 
% by uncommenting the following (adjust the "2.5in" size to make it fit
% properly)
% \setlength\titlebox{2.5in}

\begin{document}

\maketitle

\section{Introduction}

This project implements a poker player AI agent that acts as one of the two players in Limit Texas Hold’em. The fundamental idea of implementation is using reinforcement learning to build a heuristic evaluation function, which measures the expected reward of each action and make the decision that maximizes the return. Hand strength, current bet and opponent’s bet were the three components that make up the evaluation function. The evaluation function is a linear function that sum up the product of the numerical value of each component and their corresponding weight. To decide the weight of the three components, online training was conducted. The weight for each component reflects the expected influence that the component has on the final result of one round of game. The agent’s decision and final result were feed into the online training at the end of every round, and weight for each component was adjusted. Different from current bet and opponent’s bet which can be obtained directly from game state, the numerical value of hand strength was computed by evaluating hole cards and revealed community cards. A list of poker features, such as high card and pair value was selected to determine the hand strength. Similar to the evaluation function, the function that calculates hand strength is also a linear function that sums up the value of the product of weight and feature. The weight of each feature was determined from our offline training. Traditional approach of poker game agent, like listing preflop chart, is computationally expensive and space consuming. To improve the efficiency while not reducing accuracy, we choose to implement a reinforcement learning based agent. Our agent is able to learn from opponent’s behavior and adjust evaluation function according to opponent’s characteristic. To improve the accuracy of decision making, we trained the hand strength evaluation function of agent offline with large amount of data before playing game with others. In this report, we will explain how we implement the poker agent and discuss how the agent can make optimal decisions. In section 2, we will review a few approaches in past research and analysis their strength and weaknesses. In section 3, we will elaborate our agent’s play strategy, and explain the implementation details and training methods. In section 4, we will discuss the training results and show merits and limitation of our approach. Lastly, we will go into the conclusion and future direction.   

\section{Past research}
The core of the implementation of an AI poker player agent is to decide the next action it should take giving the information of the environment. Utility maximization and regret minimization are two approaches that have been well explored regarding next action decision. The most traditional method to optimize the utility is to list out results for all possible solutions in each round and select the action that gives highest utility. This method is precise and easy to implement but computational expansive. To reduce the computational effort, some researchers proposed heuristics that evaluate the return of the game and use decision trees to model the next actions that should be take. With sufficient and reasonable training, this method can yield good results. Regret minimization is another popular approach. Regret is defined to be the loss in utility taking this action suffers for not having selected the single best deterministic strategy and the single best deterministic strategy can only be known in hindsight. The approach aims to minimize the regret value. In 2015, Bowling and other 3 scientists announced that they weakly solved Heads-up limit Texas hold’em by using a technique called counterfactual regret minimization which is based on regret minimization. 
Since we discussed minimax algorithm during class and we are more familiar with the utility maximization approach, we adopted the utility maximization approach. We generate a decision tree at each step and the agent will take the action that gives the highest numeric value of the evaluation function. Evaluation function is another key element of our implementation. From past research, hand strength and pot money are two of the most widely-used factors when constructing the evaluation function. Reinforced learning is also proposed by some researchers to adjust the evaluation function based on information that cannot be got in advance, such as opponent’s behavior.  

\section{Implementation}

\subsection{Agent strategy}
To play Limit Taxas Hold'em Game, our poker agent evaluates the environment and estimates expected return of actions to make decisions. 
\begin{enumerate}
	\item When it’s the poker agent’s turn to bet, the poker agent will firstly evaluate the strength of hole cards combining revealed community cards. The hand strength is evaluated based on existing card features in hand and current betting round. The poker agent takes 18 cards features into consideration. Hand strength evaluation function is a linear combination of those 18 card feature’s value. While playing, agent uses the function settled by offline training before game to estimate the strength of cards. The function and offline training will be further elaborated in the following section. 
	\item After evaluating hand strength, agent will estimates the expected reward of the whole round of game if choosing each action (‘call’, ‘raise’ or ‘fold’). The expected reward is estimated by evaluation function, which is a linear combination of hand strength, the difference between the current bet and the opponent’s bet, the difference between the current remaining money and the money in the opponent’s stack, and the difference between my gain and the opponent’s gain:
	
	\item What is the result that you achieved? (i.e. why should the reader believe you?)
\end{enumerate}

\subsection{Evaluation function}
Generally speaking, the hand strength of the combination of hole cards and revealed community card is static. The evaluation function of hand strength is obtained by offline training and settled before the start of game. It is not affected by opponents’  behavior and will not be adjusted in the process of playing game.                       
Hand strength evaluation function is a linear combination of all features appearing in the combination of hole cards and revealed community card(s). There are 18 features that our agent use to estimate hand strength. 
\\All the 18 card features of 4 types are shown below.



\subsection{Training method}

\subsubsection{Offline training}

\subsubsection{Online training}
To enable our agent to learn from games that it have been played with the opponent, the weight for each feature in the evaluation function is adjusted to improve the accuracy of our evaluation. As mentioned in section 3.2, our evaluation function for each action calculate the expected reward of taking the action. Based on the result at the end of every round, the value of the weight vector, $\theta$ will be adjusted.
\\At the first round, pre-hardcoded base weight is used in the evaluation function. The base weight was derived from our offline training with random player (?, not sure). Monte Carlo algorithm was used.
\\At the end of each round, $\theta$ for each action will be updated inside \begin{verbatim}receive_round_result_messge() \end{verbatim}function. Stochastic gradient descent approach is used to update $\theta$.

\section{Training Result and analysis}
\subsection{Offline training result}
\subsection{Online training result}
\subsection{discussion}
Our implementation has three main advantages. Firstly, the approach we used to solve this problem is intuitive and fits the characteristic of poker game. Instead of generating game tree and running minimax algorithm to solve the problem, we use decision tree. The restriction of minimax algorithm in imperfect information game is mentioned by several researchers. Decision tree, in the other hand, requires less space and simulate well people's behavior in real life. Secondly, we took a wide range of features into consideration which makes our evaluation more complete and precise. Thirdly, by using reinforcement learning, we adjust the weight of each feature promptly.
\section{Conclusion}

Print manuscripts two columns to a page, in the manner in which these
instructions are printed. The exact dimensions for pages are:
\begin{itemize}
\item left and right margins: .75$''$
\item column width: 3.375$''$
\item gap between columns: .25$''$
\item top margin---first page: 1.375$''$
\item top margin---other pages: .75$''$
\item bottom margin: 1.25$''$
\item column height---first page: 6.625$''$
\item column height---other pages: 9$''$
\end{itemize}

All measurements assume an 8-1/2$''$ $\times$ 11$''$ page size. For
A4-size paper, use the given top and left margins, column width,
height, and gap, and modify the bottom and right margins as necessary.

\subsection{Format of Electronic Manuscript}

For the production of the electronic manuscript, you must use Adobe's
{\em Portable Document Format} (PDF). A PDF file can be generated, for
instance, on Unix systems using {\tt ps2pdf} or on Windows systems
using Adobe's Distiller. There is also a website with free software
and conversion services: {\tt http://www.ps2pdf.com/}. For reasons of
uniformity, use of Adobe's {\em Times Roman} font is strongly suggested. In
\LaTeX2e{}, this is accomplished by putting
\begin{quote} 
\mbox{\tt $\backslash$usepackage\{times\}}
\end{quote}
in the preamble.\footnote{You may want also to use the package {\tt
latexsym}, which defines all symbols known from the old \LaTeX{}
version.}
  
Additionally, it is of utmost importance to specify the American {\bf
letter} format (corresponding to 8-1/2$''$ $\times$ 11$''$) when
formatting the paper. When working with {\tt dvips}, for instance, one
should specify {\tt -t letter}.

\subsection{Title and Author Information}

Center the title on the entire width of the page in a 14-point bold
font. The title should be capitalized using Title Case. Below it, center author name(s) in a 12-point bold font. On the following line(s) place the affiliations, each affiliation on its own line using a 12-point regular font. Matching between authors and affiliations can be done using superindices. Additionally, a comma-separated email addresses list using a 12-point regular font is also allowed. Credit to a
sponsoring agency can appear on the first page as a footnote.

\subsection{Text}

The main body of the text immediately follows the abstract. Use
10-point type in a clear, readable font with 1-point leading (10 on
11).

Indent when starting a new paragraph, except after major headings.

\subsection{Headings and Sections}

When necessary, headings should be used to separate major sections of
your paper. (These instructions use many headings to demonstrate their
appearance; your paper should have fewer headings.). All headings should be capitalized using Title Case.

\subsubsection{Section Headings}

Print section headings in 12-point bold type in the style shown in
these instructions. Leave a blank space of approximately 10 points
above and 4 points below section headings.  Number sections with
arabic numerals.

\subsubsection{Subsection Headings}

Print subsection headings in 11-point bold type. Leave a blank space
of approximately 8 points above and 3 points below subsection
headings. Number subsections with the section number and the
subsection number (in arabic numerals) separated by a
period.

\subsubsection{Subsubsection Headings}

Print subsubsection headings in 10-point bold type. Leave a blank
space of approximately 6 points above subsubsection headings. Do not
number subsubsections.

\subsubsection{Special Sections}

You may include an unnumbered acknowledgments section, including
acknowledgments of help from colleagues.

Any appendices directly follow the text and look like sections, except
that they are numbered with capital letters instead of arabic
numerals.

The references section is headed ``References,'' printed in the same
style as a section heading but without a number. A sample list of
references is given at the end of these instructions. Use a consistent
format for references, such as that provided by Bib\TeX{}. The reference
list should not include unpublished work.

\subsection{Citations}

Citations within the text should include the author's last name and
the year of publication, for example~\cite{gottlob:nonmon}.  Append
lowercase letters to the year in cases of ambiguity.  Treat multiple
authors as in the following examples:~\cite{abelson-et-al:scheme}
or~\cite{bgf:Lixto} (for more than two authors) and
\cite{brachman-schmolze:kl-one} (for two authors).  If the author
portion of a citation is obvious, omit it, e.g.,
Nebel~\shortcite{nebel:jair-2000}.  Collapse multiple citations as
follows:~\cite{gls:hypertrees,levesque:functional-foundations}.
\nocite{abelson-et-al:scheme}
\nocite{bgf:Lixto}
\nocite{brachman-schmolze:kl-one}
\nocite{gottlob:nonmon}
\nocite{gls:hypertrees}
\nocite{levesque:functional-foundations}
\nocite{levesque:belief}
\nocite{nebel:jair-2000}

\subsection{Footnotes}

Place footnotes at the bottom of the page in a 9-point font.  Refer to
them with superscript numbers.\footnote{This is how your footnotes
should appear.} Separate them from the text by a short
line.\footnote{Note the line separating these footnotes from the
text.} Avoid footnotes as much as possible; they interrupt the flow of
the text.

\section{Illustrations}

Place all illustrations (figures, drawings, tables, and photographs)
throughout the paper at the places where they are first discussed,
rather than at the end of the paper. If placed at the bottom or top of
a page, illustrations may run across both columns.

Illustrations must be rendered electronically or scanned and placed
directly in your document. All illustrations should be in black and
white, as color illustrations may cause problems. Line weights should
be 1/2-point or thicker. Avoid screens and superimposing type on
patterns as these effects may not reproduce well.

Number illustrations sequentially. Use references of the following
form: Figure 1, Table 2, etc. Place illustration numbers and captions
under illustrations. Leave a margin of 1/4-inch around the area
covered by the illustration and caption.  Use 9-point type for
captions, labels, and other text in illustrations.

\section*{Acknowledgments}

The preparation of these instructions and the \LaTeX{} and Bib\TeX{}
files that implement them was supported by Schlumberger Palo Alto
Research, AT\&T Bell Laboratories, and Morgan Kaufmann Publishers.
Preparation of the Microsoft Word file was supported by IJCAI.  An
early version of this document was created by Shirley Jowell and Peter
F. Patel-Schneider.  It was subsequently modified by Jennifer
Ballentine and Thomas Dean, Bernhard Nebel, and Daniel Pagenstecher.
These instructions are the same as the ones for IJCAI--05, prepared by
Kurt Steinkraus, Massachusetts Institute of Technology, Computer
Science and Artificial Intelligence Lab.

\appendix

\section{\LaTeX{} and Word Style Files}\label{stylefiles}

The \LaTeX{} and Word style files are available on the IJCAI--18
website, {\tt http://www.ijcai-18.org/}.
These style files implement the formatting instructions in this
document.

The \LaTeX{} files are {\tt ijcai18.sty} and {\tt ijcai18.tex}, and
the Bib\TeX{} files are {\tt named.bst} and {\tt ijcai18.bib}. The
\LaTeX{} style file is for version 2e of \LaTeX{}, and the Bib\TeX{}
style file is for version 0.99c of Bib\TeX{} ({\em not} version
0.98i). The {\tt ijcai18.sty} file is the same as the {\tt
ijcai07.sty} file used for IJCAI--07.

The Microsoft Word style file consists of a single file, {\tt
ijcai18.doc}. This template is the same as the one used for
IJCAI--07.

These Microsoft Word and \LaTeX{} files contain the source of the
present document and may serve as a formatting sample.  


%% The file named.bst is a bibliography style file for BibTeX 0.99c
\bibliographystyle{named}
\bibliography{ijcai18}

\end{document}

